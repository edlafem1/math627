\documentclass[12pt]{article}

% I always modify the margins for better use of space:
% horizontal margins: 1.0 + 6.5 + 1.0 = 8.5
\setlength{\oddsidemargin}{0.0in}
\setlength{\textwidth}{6.5in}
% vertical margins: approximately  1.0 + 9.0 + 1.0 = 11.0
\setlength{\topmargin}{-0.5in}
\setlength{\headheight}{12pt}
\setlength{\headsep}{13pt}
\setlength{\textheight}{9.5in} % {650pt}
\setlength{\footskip}{24pt}

% some personal command definitions as examples:
\newcommand{\half}{\frac{1}{2}}
\newcommand{\eps}{\varepsilon}
\newcommand{\rh}{\rho}
\newcommand{\mtheta}{\vartheta}
\newcommand{\ph}{\varphi}
% this command is for partial derivatives and takes 2 input arguments:
\newcommand{\der}[2]{\frac{\partial {#1}}{\partial {#2}}}

\begin{document}

% Title page stuff:
\author{Matthias K. Gobbert}
\title{Some \LaTeX\ Introduction}
\date{\today}
\maketitle

\section{Section Title Here}
This is the beginning of a section.

The overriding principle of \LaTeX\ is that the \emph{input file}
(\verb+\emph{...}+ emphasizes the text in the argument) should be easy
to read; the system takes care of all formatting decisions!

In the following, I keep using the \verb+\verb+ command; it quotes the
text between the plus signs verbatim.

\subsection{Subsection Title Here}
Here, you see how a mathematical equation can be generated inline, for
instance $f(x) = \frac{1}{1+25 x^2}$.
The \verb+$+-symbols enclose the formula.
As a so-called displayed formula, it would look like
\begin{displaymath}
  f(x) = \frac{1}{1+25 x^2}.
\end{displaymath}
It is customary that mathematical functions are \emph{not} set in math-italics,
so \LaTeX\ has the basic ones pre-defined; you should use the commands
\verb+\cos+, \verb+\exp+, etc.\ to get $f_1(x) = \cos x$,
$f_2(x) = - e^x \sin^2 x$, etc.

Here, I use some of my commands defined above: I like $\eps = \varepsilon$
better than the default $\epsilon$. A partial derivative (with 2 arguments)
would be obtained as follows. If $f(x,y) = x^2 y^3$, then 
\begin{displaymath}
  \der{f}{x} = 2 x y^3, \quad \der{f}{y} = 3 x^2 y^2.
\end{displaymath}

\subsection{Sums and Integrals}
When you say ``capital sigma,'' you probably did not really mean $\Sigma$,
but rather a summation symbol. You would get that as in
\begin{displaymath}
  \sum_{i=0}^{\infty} r^i = \frac{1}{1 - r} \quad \mbox{for all $|r| < 1$}.
\end{displaymath}
Finally, we have
\begin{displaymath}
  \int_0^1 \sin(2 \pi x) \, dx = 0
\end{displaymath}
and
\begin{displaymath}
  \int \! \! \int f(x) g(y) \, dx \, dy = \int f(x) \, dx \,\, \int g(y) dy.
\end{displaymath}
Here, \verb+\,+ gives a small space, while \verb+\!+ forces things closer
together; you have to work on the proper spacing for integrals, as \LaTeX\
does not understand, what is going on.

\subsection{Matrices in \LaTeX}
A matrix $A \in \mathrm{R}^{m \times n}$ could be defined by
\begin{displaymath}
  A = \left( \begin{array}{ccccc}
        11     & 12     & 13     & \cdots & 1n     \\
        21     & 22     & 23     & \cdots & 2n     \\
        \vdots & \vdots & \vdots & \ddots & \vdots \\
        m1     & m2     & m3     & \cdots & mn     \\
      \end{array} \right)
\end{displaymath}
Here, the word \verb+dots+ in the commands stands for an ellipsis
(i.e., three dots) placed horizontally in the center (\verb+\cdots+),
vertically (\verb+\vdots+), or diagonally (\verb+\ddots+); what is
not mentioned is \verb+\ldots+ for horizontal dots at the baseline.
Use the baseline or central version as appropriate, for instance
\begin{eqnarray*}
  a_1, a_2, \ldots, a_n & \mbox{and not} & a_1, a_2, \cdots, a_n, \\
  a_1 + a_2 + \cdots + a_n & \mbox{and not} & a_1 + a_2 + \ldots + a_n, \\
\end{eqnarray*}

Some more comments on the matrix are needed, I suppose:
The \verb+\left(+ and \verb+\right)+ create the variable-sized parenthes
around the actual array of terms. You can also use \verb+\left[+ and
\verb+\right]+, or \verb+\left\{+ and \verb+\right\}+ in other situations.
The actual array arrangement is organized by the \verb+array+ environment;
you need the arguments \verb+ccccc+ to indicate that there are five columns
and you want the entries centered (``c''), other options are left (``l'')
and right (``r''). Notice how \verb+&+ separate columns and \verb+\\+
the rows.

\section{Further Reading}

\subsection{This document}
This document is written with the intention that you also read the source
code; indeed, many statements will only then make sense. The source of
this file can be downloaded from the homepage of this course, following
my homepage \verb+http://www.math.umbc.edu/~gobbert/+.
Furthermore, I strongly recommend the following books in the reference
list, all of which are well-written and recognized standards.

\begin{thebibliography}{9}
  \bibitem{Lamport94}
  Leslie Lamport, \emph{\LaTeX\ User's Guide and Reference Manual},
  second edition, Addison-Wesley, 1994. \emph{The} basic introduction
  to \LaTeX\ by the author of the package.
  \bibitem{Goossens94}
  Michel Goossens, Frank Mittelbach, and Alexander Samarin,
  \emph{The \LaTeX\ Companion}, Addison-Wesley, 1994.
  If you want to change \LaTeX's internal settings, this book
  is unavoidable; not needed for a novice.
  \bibitem{Higham98}
  Nicholas J. Higham, \emph{Handbook of Writing for the Mathematical Sciences},
  SIAM, 1998. A general treatise of all things pertaining to writing
  mathematics.
\end{thebibliography}

\end{document}

